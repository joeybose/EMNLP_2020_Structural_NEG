\section{Proposed Approach}
\label{sec:proposedapproach}
In this paper, we seek to explicitly use the rich graph structure surrounding a particular node when generating negative triplets. We motivate our approach based on the observation that prior work in learning word embeddings \cite{mikolov2013distributed}, where negative sampling has historically developed, lacked the richness of graph structure that is immediately accessible in the KG setting. Consequently, we hypothesize that enriching the negative sampling process with structural information in KG can yield harder negative examples, crucial to learning of effective embeddings. The preliminary step for implementing this idea is building the k-hop neighbourhood ($K$) of each node, which can be computed using the formula below: 
\begin{equation}
\label{eqn:khop}
K = A^{k} + A^{k-1}
\end{equation}
for $k>0$, where $k$ is an arbitrary number, representing the desired neighbourhood radius to be considered during sampling for a given centre node, and $A$ stands for \emph{adjacency matrix}. 

\cut{ However, since computing Eq. \ref{eqn:khop} can be costly for large KG's due to matrix multiplication, the \emph{k-hop} neighbourhood can be approximated using \emph{Random Walks (RW)} \cite{perozzi2014deepwalk} by Algorithm \ref{alg:rw}, where $\omega$ is the number of RWs, and $k$ is the number of k-hops.

\begin{algorithm}[tb]
   \caption{\small Approximating K-hop neighbourhood using RWs}
   \label{alg:rw}
  \begin{small}
\begin{algorithmic}
   \STATE {\bfseries Input:} $A, k, \omega$
   \COMMENT A: Adjacency Matrix, k: \# of k-hops, $\omega$: \# of RWs
   \STATE {$K \gets$ sparseMatrix($|A| \times |A|$)}
   \FORALL {entity $e$}
   \STATE {$K[e] \gets $} randomWalk(k, $\omega$)
   \ENDFOR
   \STATE {\textbf{return}} $K$
\end{algorithmic}
\end{small}
\end{algorithm}
}
Consequently, given a value for $k$, the \emph{k-hop} neighbourhood of an entity (node) within the graph is calculated, and the negative triples are made by selecting the head $h$ or tail $t$ entity from the \emph{k-hop} neighbourhood that satisfy this criterion:
\begin{equation}
    (h',r,t') = \{ (h',r,t)|h'\in K\} \cup \{(h,r,t')|t'\in K \}  \\
\end{equation}
with $ K \subset E$, where $K$ is the \emph{k-hop} neighbourhood of $h$ or $t$, and $E$ is the entire sample space. It should be noted that since different relation types are considered in our sampling scheme, an additional dimension is added to the adjacency (A) and \emph{k-hop} (K) matrices, resulting in adjacency and \emph{k-hop} tensors.

\subsection{Variants of SANS}

We can also extend the self-adversarial approach in \cite{sun2019rotate} using the SANS approach by restricting the negative triplet distribution in Eq.\ref{eq:dist_adv} to the \emph{k-hop} neighbourhood. In the subsequent sections, we refer to this technique as \emph{Self-Adversarial SANS}, whereas the former approach is referred to as \emph{Uniform SANS}. 

